\documentclass{article}
\usepackage{amsmath}
\usepackage{graphicx}
\begin{document}
\thispagestyle{plain}
\begin{center}
    \Large
    \textbf{Bidirectional Type Theory}
    
    \vspace{0.4cm}
    \large
    Physics 123
    
    \vspace{0.4cm}
    \textbf{Kyle Mckean}


    \begin{equation*}
      \scalebox{3}{
        $\prod_{X : \star} X \rightarrow X$
      }
    \end{equation*}
    
    \vspace{0.4cm}
    \textbf{Abstract}
\end{center}
Type Theory is a method of axiomatizing mathematics. Unlike Set Theory, Type
Theory is uniquely suited for interactions with computers allowing
for proof assistants. It also forms the theoretical foundation
of any statically typed programming language. In this talk, we'll cover
Bidirectional Type Theory a simpler and more effective approach to specify type
theories meant for computer implementation. Specifically focusing on the typing
judgements that axiomatize the theory. If time permits we will construct
the natural numbers using a type of well founded trees.
\end{document}